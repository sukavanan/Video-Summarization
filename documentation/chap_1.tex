
\chapter{Introduction} 
\label{Chapter1} 
\lhead{Chapter 1. \emph{Introduction}} 

    With the rapid proliferation of electronic gadgets, there's a remarkable growth in the amount of textual and non-textual data generated by every individual. Further, the sudden surge in the availability of multiple video-on-demand platforms like Hotstar, Amazon Prime Video, Netflix, and YouTube has led to an exponential increase in the amount of visual content generated. 
    
    About 500 hours of video is uploaded every minute to YouTube, and this rate of growth is only accelerating. In addition to this, users watch upwards of 1 billion hours of content every day \cite{Youtube_2019}. So far, the research on multimedia data has been majorly on extracting the statistics and syntactic details of data. There are plenty of methods available to extract the objective content from the data like recognizing objects or scenes, finding the motion of the objects, but there is not much contribution on extracting the semantic context of a video. In recent times, the research interest has shifted towards unveiling the semantic context of the data. Many deep learning approaches concentrate on extracting the semantic context of the data such as emotion recognition, summarization, etc. 
	
	Though video-sharing and streaming platforms like YouTube and Netflix\cite{netflix} have advanced recommendation systems powered by the behavior of users and their dwell time on videos, the downside is that users are becoming selective of what they see and prefer to know the substance of the content beforehand \cite{youtube}. Thus, paired with the fact that users are spending increasingly less time watching videos it becomes important to convey the content as quickly as possible with as little filler content as possible. This has sparked a rise in approaches to "summarize" content, aiming to reduce the length of the video besides retaining the informativeness and perception quality of the video. Video summarization is categorized into static and dynamic approaches. Static approaches are keyframe based that results in a storyboard summary. But the temporal evolution of the video is lost in the process of extracting the keyframes which highlight the important scenes of the video. On the contrary, dynamic approaches preserve the temporal evolution besides skimming the highlights from the video. 
	
	Summarizing a long sequenced video has many applications in scenarios where time is a critical resource. As summarized video is expected to contain the essence of the whole video, it is highly beneficial for academicians on educational platforms to revisit a topic. Further, summarizing a video ensures that the cognitive load on the minds of learners is minimized. Work done by Brame \textit{et. al.} in \cite{brame2015effective} describes multiple types of mental effort necessary to assimilate video-form educational content. This paper discusses the additional unwanted cognitive impact called \textit{Extraneous load} which makes video understanding harder due to malformed lessons, or poorly structured content. 
	
	In another paper by Guo \textit{et. al.} which analyzed over \textbf{6.9 million video sessions} over four MOOC courses on eDX to map student engagement with length of the videos. Their findings also point to the fact that shorter videos are more engaging with normalized engagement time close to 100\% till the 6 minutes to 9 minute range and progressively falling off thereafter \cite{guo2014video}.

	
	Thus video summarization serves a very important purpose for educational video comprehension. It will also be useful for general public to be informed about the summarized News headlines for a week or a month. Now, the complexity lies in knowing what constitutes importance in a video. The importance and topic relevance are subjective and differs from person to person. Moreover, another challenging fact is the highly diversified unstructured content which makes the semantic understanding difficult.